\documentclass[12pt,letterpaper,spanish]{article}
\usepackage[latin1]{inputenc}
\usepackage[spanish]{babel}
\usepackage{graphicx}

\begin{document}

\begin{center}
{\huge {\rm {\em Esquema XML}}} \\
\vspace{5mm}
{\Large Alejandro Hern\'andez Hern\'andez}\\
\end{center}
\vspace{5mm}

El documento {\em XML} visto en clase, la carta, tiene le siguiente formato: 
\begin{verbatim}

<?xml version="1.0"?>
<!DOCTYPE carta SYSTEM "carta.dtd">
<carta>
 <contacto tipo="remitente">
  <nombre>Alejandro Hernandez</nombre>
  <direccion1>Vicente Guerrero #1135</direccion1>
  <direccion2>Av. Juarez No. 235</direccion2>
  <ciudad>CDMX</ciudad>
  <estado>CDMX</estado>
  <cp>07324</cp>
  <telefono>5523146598</telefono>
  <bandera genero="M"/>
 </contacto>

 <saludo>Estimado amigo</saludo>
 <parrafo>la wea shidori de game alv :v</parrafo>
 <parrafo>alguna frase de naruto (de relleno pues :v)</parrafo>
 <cierre>grax men por leer esto xD</cierre>
 <firma>yo merengues</firma>
</carta>

\end{verbatim}
\vspace{1mm}
\small{\em{nota: lo importante es la estructura y elementos, no el contenido entre los elementos}}
\newpage
Entonces, podemos definir nuestro {\em XML Schema} como:
 
\begin{verbatim}

<?xml version="1.0"?>
<schema xmlns: "http://www.w3.org/2001/XMLSchema">
<element name "carta">
<complexType name="contacto">
 <sequence>
  <element name="nombre" type="string>
  <element name="direccion1" type="string">
  <element name="direccion2" type="string">
  <element name="ciudad" type="string">
  <element name="estado" type="string">
  <element name="cp" type="int">
  <element name="telefono" type="long">
 </sequence>
 <element name="saludo" type="string">
 <element name="parrafo" type="string">
 <element name="cierre" type="string">
 <element name="firma" type="string">
</complexType>
</schema>

\end{verbatim}

%termina documento
\end{document}
